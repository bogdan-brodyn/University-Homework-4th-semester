\documentclass{article}
\usepackage[T2A]{fontenc}
\usepackage[russian]{babel}
\usepackage{amsmath}

\title{Normalization}
\author{bogdan-brodyn}
\date{March 13, 2025}

\begin{document}

\maketitle
\thispagestyle{empty}

\begin{align}
&((\lambda a.(\lambda b.b\ b)\ (\lambda b.b\ b))\ b)\ ((\lambda c.(c\ b))\ (\lambda a.a)) \rightarrow_\beta \\
&(\lambda b.b\ b)\ (\lambda b.b\ b)\ ((\lambda c.(c\ b))\ (\lambda a.a)) \rightarrow_\beta \\
&(\lambda b.b\ b)\ (\lambda b.b\ b)\ ((\lambda c.(c\ b))\ (\lambda a.a))
\end{align}

Согласно теореме Карри о нормализации, нормальная стратегия приведёт $\lambda$-терм к нормальной форме, если такая существует. Сокращение самого левого внешнего редекса при переходе от (2) к (3) не оказало эффекта на $\lambda$-терм. Таким образом приходим к выводу, что нормальной формы нет.

\end{document}
